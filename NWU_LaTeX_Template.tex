\documentclass[a4paper, 11pt]{article} %Here you can change the font size of the entire document


%\\------------------------------------------------------\\
%\\ Packages needed for the operation of the template
%\\ DO NOT DELETE THIS
\usepackage{sectsty} %Package needed to format the section headings. This package is responsible for the \underbar and \underline errors
\usepackage{graphicx} %Package for importing graphics (pics)
\usepackage{ragged2e} %Package for justifying text
\usepackage{tikz} %Package for advanced image positioning
\usepackage{xcolor} %Package for changing the color of text
\usepackage{etoolbox} %Package to change abstract title to be at the top of the page
\patchcmd{\abstract}{\null\vfil}{}{}{} %Abstract patch: do nog delete
\usepackage{tocloft} %Package to properly format Table of Contents
\setlength{\cftbeforetoctitleskip}{0pt}
\setlength{\cftaftertoctitleskip}{0pt}
\usepackage[a4paper, margin=2.54cm]{geometry} %Specifies the size of text area on the paper
\usepackage{hyperref} %Package to make the Table of Contents use hyperlinks
\usepackage{etoolbox} %Package to remove abstract indent
\ifundef{\abstract}{}{\patchcmd{\abstract}%
    {\quotation}{\quotation\noindent\ignorespaces}{}{}}
\usepackage{lipsum}    

%\\------------------------------------------------------\\


%\\------------------------------------------------------\\
%\\Add user packages here\\
%\usepackage{natbib} %Package to use Harvard style bibliography, comment out if using IEEE style
%\usepackage{showframe} %Helper package to show document outlines
%\\------------------------------------------------------\\


%\\------------------------------------------------------\\
%\\ Font Specifier
%\\ Comment out this section if you want standard Times New Roman Font
\usepackage{helvet} %Arial Font
\renewcommand{\familydefault}{\sfdefault} %Command needed to make font arial
%\\------------------------------------------------------\\


%\\------------------------------------------------------\\
%\\ Formatting of headings
%\\ Here you can specify the size and color of all headings
\sectionfont{\LARGE }
\subsectionfont{\Large }
\subsubsectionfont{\large }
\renewcommand\abstractname{\LARGE Abstract} %To make the abstract heading purple
\renewcommand{\contentsname}{\center \Large Table of Contents} 
\renewcommand{\listfigurename}{\center\ \Large List of Figures}
\renewcommand{\listtablename}{\center \Large List of Tables }      
\renewcommand\thesection{\arabic{section}} %Specifies the numbering of sections
%\renewcommand{\refname}{Bibliography} %Change this to your preferred heading (i.e. Bibliography, List of References etc)
%\\------------------------------------------------------\\


%\\------------------------------------------------------\\
%\\ Page formatting
\setlength{\parindent}{0in} 
\setlength{\parskip}{8pt} %The space between paragraphs
\renewcommand{\baselinestretch}{1.08} %The line spacing
\graphicspath{ {cover/} } %Directory for cover imag
%\\------------------------------------------------------\\


%\\------------------------------------------------------\\
%\\ NWU RGB Purple Color definition
%\\ DO NOT CHANGE THIS
\definecolor{Puk_Pers}{RGB}{108, 61, 145}
%\\------------------------------------------------------\\


%\\------------------------------------------------------\\
%\\ Reformatting of Abstract
\let\oldabstract\abstract
\let\oldendabstract\endabstract
\makeatletter
\renewenvironment{abstract}
{\renewenvironment{quotation}%
               {\list{}{\addtolength{\leftmargin}{-9mm} % change this value to add or remove length to the the default
                        \listparindent 0em%
                        \itemindent    \listparindent%
                        \rightmargin   \leftmargin%
                        \parsep        \z@ \@plus\p@}%
                \item\relax}%
               {\endlist}%
\oldabstract}
{\oldendabstract}
\makeatother
%\\------------------------------------------------------\\

\begin{document}

\begin{titlepage}

%\\------------------------------------------------------\\
%\\This is for the purple splash
\begin{tikzpicture}[remember picture, overlay, transform shape]
\node [anchor=north west, inner sep=0pt]
    at (current page.north west)
    {\includegraphics[width=\paperwidth]{NWU_Purple_Cover.png}};
\end{tikzpicture}	
%\\------------------------------------------------------\\


%\\------------------------------------------------------\\
%\\ Cover material content
\center
\vspace{4cm}

{\Huge PROJECT TITLE\par}

\vspace{2cm}
	
{\Large\itshape INITIALS AND SURNAME\par}
	
\vspace{2cm}
	
{\LARGE DOCUMENT TYPE HERE  e.g. Practical Report\par}
		
\vspace{5cm}
	
\begin{flushleft}

\large	Supervisor: SUPERVISOR NAME\par
\large	Date: \today  \par %Do not remove the \today command. This automatically changes the date on the cover page to the current date
\large	Student No: STUDENT NO\par
\end{flushleft}

\end{titlepage}
%\\------------------------------------------------------\\

%\\------------------------------------------------------\\
%\\Abstract Section
\thispagestyle{plain} 
\pagenumbering{roman} 
\begin{abstract}
\addcontentsline{toc}{section}{Abstract}

\justify

\lipsum[1]


\end{abstract}


\newpage %Use newpage here instead of \pagebreak

\tableofcontents

\listoffigures

\listoftables

\pagebreak

\pagenumbering{arabic}

%\\------------------------------------------------------\\
%\\Content Section
\section{Section}
The quick brown fox jumps over the lazy dog.

\subsection{Subsection 1}
The quick brown fox jumps over the lazy dog.

\subsubsection{SubSubsection 1}
The quick brown fox jumps over the lazy dog.

\subsection{Subsection 2}
The quick brown fox jumps over the lazy dog.

\section{Section 2}
The quick brown fox jumps over the lazy dog.


%\\------------------------------------------------------\\
%\\Common documents extras and how to format them in LaTeX

%\\Figure
\begin{figure}[h!]
\center
\includegraphics[scale=0.5]{NWU_Purple_Cover.png}
\caption{CAPTION}
\label{LABEL_NAME} %LABEL MUST ALWAYS BE AFTER CAPTION
\end{figure}

With refernce to figure \ref{LABEL_NAME} %This is how you reference a figure/table/section/subsection etc Anything that has a \label is referred to in this way

%Numbered Equation 
%This is to insert a numbered equation
\begin{equation} 
x^2 + 4x + 5 
\label{EQUATION NAME}
\end{equation}

%Inline Mathematics
$ x + 1 = 12 $ %use the $ sign for maths you want in your paragraphs

%Tables
%For a detailed guide to tables use https://www.overleaf.com/learn/latex/Tables
%To generate tables easily make use of https://tablesgenerator.com/
%If using the generator above remember to add caption, label and \center to the table code just underneath the \begin{table}
%e.g.
\begin{table}[h!]
\center
\caption{CAPTION}
\vspace{0.2cm}
\label{LABEL NAME}
\begin{tabular}{|l|l|}
\hline
a & b \\ \hline
c & d   \\ \hline
\end{tabular}
\end{table}

%\cite{BIBLIOGRAPHY_ITEM} - Citation example

%\\------------------------------------------------------\\
%\\ Uncomment this section for bibliography however make sure that the filename in \bibliography{} matches the actual filename of the .bib file
%\\Use Zotero with the Better BibLaTeX add-on installed for the easiest bibliography setup
%\addcontentsline{toc}{section}{Bibliography} %References/Bibliography has to be manually added to ToC
%\bibliographystyle{IEEEtran} %- IEEE Style
%\bibliographystyle{humannat} %- Harvard style
%\bibliography{BIB FILE NAME}
\end{document}
